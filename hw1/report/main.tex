\documentclass[11pt]{article}
\usepackage[margin=1in]{geometry}
\usepackage{amsmath,amsthm,amssymb}
\usepackage{algorithm}
\usepackage{algpseudocode}
\usepackage{enumerate}
\usepackage{fancyhdr}
\usepackage{hyperref}
\usepackage{graphicx}
\usepackage{parskip}
\usepackage{bookmark}
\usepackage{float}

\usepackage{listings}
\usepackage{xcolor}

\lstset{
    language=XML,
    basicstyle=\ttfamily\small,
    keywordstyle=\color{blue},
    commentstyle=\color{green!60!black},
    stringstyle=\color{red},
    showstringspaces=false,
    breaklines=true,
    frame=single,
    numbers=left,
    numberstyle=\tiny\color{gray}
}


% 设置页眉
\pagestyle{fancy}
\lhead{CS5800: Algorithms}  % 课程名称
\chead{Homework 2}          % 作业编号
\rhead{Fall 2025}          % 学期
\lfoot{}
\cfoot{\thepage}
\rfoot{}

% 定义theorem环境
\newtheorem{theorem}{Theorem}
\newtheorem{lemma}[theorem]{Lemma}
\newtheorem{claim}[theorem]{Claim}

% 标题信息
\title{CS7180: Sptp in AI: Vibe Coding\\Homework 1}  % 课程名和作业编号
\author{Qingyang Zhang\\
\small Student ID: 002591724  % 在这里填写你的学生ID
}
\date{Due: Tuesday, Feburary 2}
\begin{document}
\maketitle

\section{Prompts making}
The most of content of prompts are in a text file with XML labels.
\subsection{General Prompt}
This prompt is used to set limits:
\begin{lstlisting}
Please use the following rules for further development. If there is anyting 
unclear about the process, ask me about it. Do not write code until I tell you
 to do so
\end{lstlisting}
We use this prompt before we upload prompt document. This is essential as those questions serve as one of the bases for refining prompts

\subsection{Prompt Document(Minimal Prompt Template)}
The prompt document usually has following structures:
\begin{lstlisting}
<system>
You are a {role here} with {technical depth expectation for LLM. for example, with 10 years of experience}
You are developing {content here}
</system>

\end{lstlisting}

\section{Prompt Iteration}
\begin{figure}[htbp]  % h=here, t=top, b=bottom, p=page
    \centering
    \includegraphics[width=0.8\textwidth]{imgs/iter.png}
    \caption{Iteration Process}
\end{figure}

The iteration of prompt file has following steps:
\begin{enumerate}
    \item Write initial prompt document from scratch.
    \item Upload initial prompt to LLM along with query template(in which we will require LLM to ask about implemtation detail)
    \item Repeat: Answer questions \& add more prompts util:1) We are satisfied about the plan LLM gives us; 2)LLM thinks the plan works and stop giving out questions.
    \item Require LLM proceeds to implementation.
    \item Validate code from three aspects: \begin{itemize}
        \item Prompts and initial expectation about this code(for example, modular TypeScript class);
        \item Additional expectation after seeing the code.(for example, I want them in separate files/ I noticed that object-oriented is to sophisticated and not needed.)
        \item Test result
    \end{itemize}
    \item Repeat: Require LLM to create new versions(for example, v1.1 to fix v1.0) of artifact to fix existing problems. If new versions created, require LLM to creat development log based on a give template.
    \item When the code is satisfying enough, the conversation stops. Then imporve prompt document and begin a new iteration(Create v2.0 with imporved prompts for v1.0).
\end{enumerate}

\section{Summarization}

\subsection{What Makes Good Prompts}
Based on observation, most components that make good prompts have already been covered in the lectures, but the specific direction and content require refinement.

\subsubsection{Detailed Description about Output}

Possibly, we should always clarify whether we want a piece of code or files. Specifically, we need to give out file names to encourage Claude to use artifacts.\textbf{If we don't give out specific file name, Claude is less likely to use artifact, and will give out code in the chatting window}
It will also ingore any of your command that requires using an artifact, as it thinks that the content in the chatting window is an artifact.

\subsubsection{Repeat Limitations \& Ask, Don't Describe}

Claude sometimes \textbf{ignore requirements mentioned at the beginning of the conversation, even though the conversation is short.}
And, if we just give out description(like test result) without a specific requirement, this ignorance is more likely to happen.
For example, \texttt{"Don't code until I tell you"} is usually ignored. This may be a result of Anthropic emphasizing model diligence, and we sometimes need to impose limits on this.

\subsubsection{Describe the Code's Execution/Compilation Process}

\appendix

\section{Links}
\textbf{GitHub Repository}: \url{https://github.com/YoungZ2357/CS7180-SptpAI-VibeCoding-HW}



\textbf{Challenge 1}
\begin{itemize}
    \item Version1: \url{https://claude.ai/share/7fa640bc-53f0-421a-89e4-d7a1892d0550}
    \item Version2: \url{https://claude.ai/share/b478f1e3-ba38-4672-8d69-ed1f7a9da809}
\end{itemize}

\textbf{Challenge 2}
\begin{itemize}
    \item Version1: \url{https://claude.ai/share/9ecef542-1d4c-464b-bfdc-4969fba770cc}
    \item Version2: \url{https://claude.ai/share/d58574ea-41ee-45ce-afc2-6d35896cbfc5}
\end{itemize}
\end{document}